\documentclass[12pt]{article}
\usepackage[UTF8]{ctex}
\usepackage{nsfc}

\newcommand{\query}[1]{\textcolor[rgb]{0,0.6,0}{Query: #1}}
\newcommand{\response}[1]{\textcolor[rgb]{0,0,0.6}{Response: #1}}
\graphicspath{{figures/}}


\begin{document}

\section{主要问题}

\query{所研究的输入数据特性没有明确,我们做的是双目RGB+光谱相机的数据,所有的要解决的问题表述中,都要体现出来融合定位这一问题;目前所有的研究表述中都没有提及光谱图像,没有分析光谱图像的特点,也没有提及如何融合提高6D姿态估计的精度。    }

\response{回复}

\query{没有结合“危险品”这个主题,在所有的内容表述中都没有分析我们要解决的问题和“危险品”之间的关系;}

\response{回复}

\query{目前的本子撰写深度欠缺比较多,从语句表述上看,还是外行的泛泛的水平,没有深入到问题点上。}

\response{回复}

\query{本子基本没有和“光谱”关联起来,如果没有光谱,我们很多已有基础都无法支撑 }

\response{回复}

\section{细节问题}

\query{ }

\response{回复}

\end{document}